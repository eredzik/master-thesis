% Template authors: Bogumił Kamiński and Michał Jakubczyk

\documentclass[12pt,a4paper,twoside,openany]{book}
\usepackage[T1]{fontenc}
\usepackage[utf8]{inputenc}
\usepackage[polish]{babel}
\usepackage{graphicx}
\usepackage{times}
\usepackage{indentfirst}
\usepackage[left=3cm,right=2cm,top=2.5cm,bottom=2.5cm]{geometry}
% \usepackage{natbib}
\usepackage{enumitem}
\usepackage{color}
\usepackage{soul}
\usepackage{tikz}
\usepackage{url}
\usepackage{todonotes}
\usepackage{verbatim}
\setlist{itemsep=0pt}
\setlist{nolistsep}
\frenchspacing
\linespread{1.3}
\addto\captionspolish{%
\renewcommand*\listtablename{Spis tabel}
\renewcommand*\tablename{Tabela}
}
\usepackage{titlesec}
\titlelabel{\thetitle.\quad}

\frenchspacing

\begin{document}

\begin{center}
\includegraphics[scale=0.3]{sgh_full.png}

\vspace{1cm}

% tu i dalej fbox należy usunąć i wpisać odpowiednią wartość
Studium magisterskie
\end{center}

\vspace{1cm}

\noindent Kierunek: Analiza danych Big Data

\noindent Specjalność: Brak

\vspace{1cm}

{
\leftskip=10cm\noindent
Emil Redzik\newline
Nr albumu: 64121

}

\vspace{2cm}

\title{Wzór pracy dyplomowej SGH}
\makeatletter

\begin{center}
\LARGE\bf
\fbox{\@title}
\end{center}

\vspace{2cm}

{
\leftskip=10cm\noindent
Praca \fbox{magisterska/licencjacka}\newline 
napisana w~Instytucie Ekonometrii\newline
pod kierunkiem naukowym\newline
\fbox{promotor}

}

\vfill

\begin{center}
Warszawa, 2021
\end{center}
\thispagestyle{empty}

\clearpage
\thispagestyle{empty}
\mbox{}
% druga strona będzie pusta, ponieważ drukujemy dwustronnie
% a mbox jest po to, żeby ta strona się pokazała
% od procenta robimy komentarze
\clearpage

\tableofcontents

\clearpage

\chapter*{Uwagi techniczne, skasuj rozdział po uwzględnieniu}

Wymagane jest wgranie pracy do systemu Overleaf (\url{https://www.overleaf.com/}) i udostępnienie promotorowi z~prawem zmian.

Pytania do promotora w~tekście proszę zadawać \todo[inline]{o tak}, tj.~stosując \verb!\todo[inline]{o tak}!.

Po kropkach niekończących zdania (np.~po `np.', `tj.', `itd.', itd.) stawiamy znak tyldy, tj.~\verb!~! (żeby zmniejszyć odstęp). Wykorzystujemy także tyldę między odwołaniem i~numerem, np.~`Tabela~3' (oczywiście odwołanie robione automatycznie przez \verb!\ref{}!). I wreszcie, stosujemy tyldę po pojedynczych literach (np.~`i', `a'), bo to niełamliwa spacja i~gwarantuje brak takich liter na końcu linii.

Przykłady odwołań źrółowych:
% \begin{itemize}
% \item \citep{dolan2000}, \citet{dolan2000}, \defcitealias{dolan2000}{Dolana (2000)}\citetalias{dolan2000};
% \item \citep{wakker1999}, \citet{wakker1999}, \defcitealias{wakker1999}{Wakkera i Zhanka (2000)}\citetalias{wakker1999};
% \item \citep{drummond1997}, \citet{drummond1997}, \defcitealias{drummond1997}{Drummonda et al.~(2000)}\citetalias{drummond1997}.
% \end{itemize}

Co warto przeczytać o~\LaTeX:
\begin{enumerate}
\footnotesize % tak zmieniamy rozmiar fontu
\item \url{http://www.tex.ac.uk/ctan/info/gentle/gentle.pdf},
\item \url{ftp://sunsite.icm.edu.pl/pub/CTAN/info/lshort/english/lshort.pdf},
\item \url{http://paws.wcu.edu/tsfoguel/tikzpgfmanual.pdf},
\item \url{https://www.overleaf.com/latex/learn/free-online-introduction-to-latex-part-1}.
\end{enumerate}

\chapter{Wprowadzenie}

Celem niniejszej pracy jest \ldots

\chapter{Kolejny rozdział}
Wstęp
\section{Podstawowe pojęcia}
\begin{itemize}
\item Definicja grafu \cite{gnn.review}
\item Grafy skierowane vs nieskierowane
\item Grafy homogeniczne vs heterogeniczne
\item Grafy dynamiczne vs statyczne
\item Zastosowania grafów (node classification, edge classification, node 
prediction, edge prediction, node embedding etc.)
\item Rodzaje uczenia (supervised, semi-supervised, unsupervised)

\end{itemize}

\section{Grafowe sieci neuronowe (GNN)}
\begin{itemize}
    \item 
\end{itemize}




\chapter{I następny}
\label{sec:nast}

Bieżący rozdział (czyli rozdział~\ref{sec:nast}) zawiera odwołanie do samego siebie. Stosujmy odwołania automatyczne.

Dodatkowo w niniejszym rozdziale zamieszczono przykładową tabelę (uwaga --- \LaTeX\ umiejscawia ją dynamicznie).

% warto w kodzie poniżej zwrócić uwagę na prawidłowy sposób formatowania liczb z przecinkiem jako znakiem dziesiętnym
% a dodatkowo na wyrównywanie liczb ułatwiające porównywanie


\begin{table}[h]
\centering
\caption{Przykładowa tabela.}
\label{tab:przyklad}
\footnotesize
\begin{tabular}{|l|r|r|}
\hline
Wariant & $N=5$ & $N=10$\\
\hline
A & $2$ & $3\phantom{{,}1}$ \\
B & $5$ & $3{,}1$ \\

\hline
\end{tabular} 
\end{table}

\clearpage

\chapter{Matematyka}

Wzory proste umieszczamy w tekście $a^2+b^2=c^2$, wzory bardziej skomplikowane --- poza nim:
\begin{equation}
\sum_{n=1}^{+\infty}\frac{1}{n^2}=\frac{\pi^2}{6}.
\end{equation}

\section{Pierwszy podrozdział}

\begin{figure}[h]
  \centering
  \begin{tikzpicture}[scale=4]
    \draw[->] (0,0) -- (1.4,0) node[anchor=west] {$n$};
    \draw[->] (0,0) -- (0,1.35) node[anchor=south] {Iloraz};
    \foreach \y in {0, 0.25,0.5,0.75,1, 1.25}
      \draw (0.03, \y) -- (-0.03, \y) node[anchor=east] {$\y$};
     \foreach \x in {0, 2, 4, 6, 8, 10, 12}
       \draw ({\x/10}, 0.03) -- ({\x/10}, -0.03) node[anchor=north] {$\x$};

       \draw (0.200000,1.164983) circle[radius=0.02];
       \draw (0.300000,1.176768) circle[radius=0.02];
       \draw (0.400000,0.877412) circle[radius=0.02];
       \draw (0.500000,0.456929) circle[radius=0.02];
       \draw (0.600000,0.351427) circle[radius=0.02];
       \draw (0.700000,0.294229) circle[radius=0.02];
       \draw (0.800000,0.246016) circle[radius=0.02];
       \draw (0.900000,0.224231) circle[radius=0.02];
       \draw (1.000000,0.168351) circle[radius=0.02];
       \draw (1.100000,0.191722) circle[radius=0.02];
       \draw (1.200000,0.173029) circle[radius=0.02];
       \draw (1.300000,0.176309) circle[radius=0.02];

       \fill (0.200000,1.225646) circle[radius=0.02];
       \fill (0.300000,0.986035) circle[radius=0.02];
       \fill (0.400000,0.945566) circle[radius=0.02];
       \fill (0.500000,0.688660) circle[radius=0.02];
       \fill (0.600000,0.556635) circle[radius=0.02];
       \fill (0.700000,0.483555) circle[radius=0.02];
       \fill (0.800000,0.443744) circle[radius=0.02];
       \fill (0.900000,0.413361) circle[radius=0.02];
       \fill (1.000000,0.344671) circle[radius=0.02];
       \fill (1.100000,0.374543) circle[radius=0.02];
       \fill (1.200000,0.358811) circle[radius=0.02];
       \fill (1.300000,0.362736) circle[radius=0.02];
  \end{tikzpicture}
  \caption{Oto przykładowy, skomplikowany rysunek. Często będzie łatwiej. \label{fig:algcomparison}}
\end{figure}

Każdy rysunek przywołujemy w tekście, także rysunek \ref{fig:algcomparison}. Można też ładować obrazy, jak pokazano na stronie tytułowej.

\section{Drugi}

\section{Trzeci}

\clearpage

\chapter{Podsumowanie}



\clearpage
\addcontentsline{toc}{chapter}{Bibliografia}
% \begin{thebibliography}{99}
% \setlength{\itemsep}{0pt}%
% \bibitem[Dolan(2000)]{dolan2000} Dolan P.~(2000), The Measurement of (...), w: A.J.~Culyer, J.P.~Newhouse (ed.), Handbook of Health Economics, Volume 1B, Elsevier, s.~1723--1760
% \bibitem[Wakker i Zhank (1999)]{wakker1999} Wakker P., H.~Zank (1999), A Unified Derivation (...), \textit{Journal of  Mathematical Economics}, 32, s.~1-19
% \bibitem[Drummond et al.~(1997)]{drummond1997} Drummond M.F., B.J.~O'Brien, G.L.~Stoddart, G.W.~Torrance (1997), Methods for Economic Evaluation (...), Oxford University Press
% \end{thebibliography}

\clearpage
\addcontentsline{toc}{chapter}{Spis rysunków}
\listoffigures

\clearpage
\listoftables
\addcontentsline{toc}{chapter}{Spis tabel}

\appendix
\chapter*{Kody źródłowe}
\addcontentsline{toc}{chapter}{Kody źródłowe}

\section*{Analiza 1}
\begin{verbatim}
x <- 1:10
y <- x + rnorm(10)
summary(lm(y ~ x))
\end{verbatim}

\section*{Analiza 2}
\begin{verbatim}
x <- rnorm(10000)
plot(density(x))
\end{verbatim}

\clearpage

\chapter*{Streszczenie}
\addcontentsline{toc}{chapter}{Streszczenie}

Tutaj piszemy streszczenie. Między 200 a 350 słów. Nie jest omówieniem struktury pracy.
Zawiera cel, metodę, dane, wyniki, wnioski.
Nie ma odwołań źródłowych, list, wykresów, tabel. Ma być zrozumiałe dla osoby nieczytającej pracy.
% \cite{dolan2000}
\cite{Scarselli2009TheGN}
\cite{Hamilton2017InductiveRL}
abc
\clearpage

\bibliographystyle{plain}
\bibliography{refs}


\end{document}
