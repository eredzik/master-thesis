\documentclass[12pt,a4paper,twoside,openany]{book}
\usepackage[T1]{fontenc}
\usepackage[utf8]{inputenc}
\usepackage[polish]{babel}
\usepackage{graphicx}
\usepackage{times}
\usepackage{indentfirst}
\usepackage[left=3cm,right=2cm,top=2.5cm,bottom=2.5cm]{geometry}
% \usepackage{natbib}
\usepackage{enumitem}
\usepackage{color}
\usepackage{soul}
\usepackage{tikz}
\usepackage{url}
\usepackage{todonotes}
\usepackage{verbatim}
\setlist{itemsep=0pt}
\setlist{nolistsep}
\frenchspacing
\linespread{1.3}
\addto\captionspolish{%
\renewcommand*\listtablename{Spis tabel}
\renewcommand*\tablename{Tabela}
}
\usepackage{titlesec}
\titlelabel{\thetitle.\quad}

\frenchspacing

\begin{document}

\begin{center}
  \includegraphics[scale=0.3]{sgh_full.png}

  \vspace{1cm}

  % tu i dalej fbox należy usunąć i wpisać odpowiednią wartość
  Studium magisterskie
\end{center}

\vspace{1cm}

\noindent Kierunek: Analiza danych Big Data

\noindent Specjalność: Brak

\vspace{1cm}

{
  \leftskip=10cm\noindent
  Emil Redzik\newline
  Nr albumu: 64121

}

\vspace{2cm}

\makeatletter

\begin{center}
  \LARGE\bf
  Grafy
  % TODO: Dodać nazwę

\end{center}

\vspace{2cm}

{
  \leftskip=10cm\noindent
  Praca magisterska\newline
  napisana w NAZWAINSTYTUTU\newline
  % TODO: Dodać nazwę instytutu
  pod kierunkiem naukowym\newline
  Sebastiana Zająca
  % TODO: Dodać promotora

}

\vfill

\begin{center}
  Warszawa, 2021
\end{center}
\thispagestyle{empty}

\clearpage
\thispagestyle{empty}
\mbox{}
% druga strona będzie pusta, ponieważ drukujemy dwustronnie
% a mbox jest po to, żeby ta strona się pokazała
% od procenta robimy komentarze
\clearpage

\tableofcontents

\clearpage


\chapter{Cel pracy}

Celem niniejszej pracy jest przedstawienie nowoczesnych metod modelowania danych reprezentowanych w formie grafu za pomocą sieci neuronowych.
Szczególny nacisk położony został na zastosowanie tych metod do predykcji wierzchołków, ale przedstawione metody mogą zostać wykorzystane do szerszej gamy zadań.
\indent

Wraz z upowszechnieniem się w ostatnich latach
wykorzystania kart graficznych do wydajnych obliczeń naukowych pojawiły się bardzo duże postępy w rozwoju sztucznych sieci neuronowych.
Ich zastosowanie staje się coraz bardziej powszechne w biznesie i nauce,
jednak większość tych metod opiera się o dane ustrukturyzowane (tabelaryczne).
Niestety te metody nie pozwalają na uwzględnienie w ramach modelu relacji pomiędzy
poszczególnymi encjami danych w naturalny sposób i wymagają albo
zmiany poziomu reprezentacji danych poprzez agregację albo zrezygnowanie z tych danych.
Tworzenie różnego rodzaju agregacji nie pozwala na wykorzystanie kluczowej informacji jaką są same relacje.
W wykorzystaniu tego typu danych pomagają grafowe sieci neuronowe.
Znajdują one zastosowania m. in. w badaniach społecznych \cite{DBLP:journals/corr/abs-1807-05560},
badaniach struktur molekularnych \cite{duvenaud2015convolutional}
\section{Założenia projektu}

\chapter{Podstawowe pojęcia}

Grafem nazywamy strukturę danych która jest zdefiniowana jako
$G = (V, E)$ gdzie $v_i\in V $ to
zbiór wierzchołków i $E = (v_i, v_j) \in V \times V$ to zbiór krawędzi pomiędzy nimi.
\cite{waikhom2021graph}
Krawędzie mogą być skierowane co będzie oznaczać kierunek relacji. Graf w którym krawędzie są skierowane nazywamy grafem
skierowanym. Do każdego z wierzhołków $V$ mogą być przypisane wektory cech $x_i$.

\section{Warianty grafów}
\subsection{Skierowanie}
Gdy posiadamy informację o kierunkach relacji w grafie to możemy mówić o grafie skierowanym.
Duża liczba grafów może być przedstawiona jako graf skierowany -
naturalnym przykładem będzie zbiór przelewów pomiędzy kontami w banku.
Pieniądze w ramach przelewu płyną w konkretnym kierunku i zagubienie tej informacji by całkowicie zmieniło
interpretację danych w tym grafie. Przykładem grafu nieskierowanego będzie
sieć społeczna znajomych - relacja przyjaźni między dwoma osobami jest obustronna i nie ma konkretnego, jasno określonego kierunku.

\subsection{Homo/heterogeniczność}
O grafie herogenicznym mówimy gdy w ramach grafu posiadamy różne rodzaje wierzchołków.
W przykładzie o przelewach w ramach banku możemy rozróżnić różne rodzaje
kont bankowych - konta techniczne banku, konta firm,
konta w innych banków, konta osób fizycznych.

\subsection{Dynamiczność}
Graf jest dynamiczny gdy jego zawartość zmienia się w czasie.
W ramach tych zmian mogą pojawić się nowe lub być usunięte wierzchołki i krawędzie.

\subsection{Attributed graph}
Krawędzie posiadają cechy

\section{Klasyfikacja zadań na grafach}
\subsection{Poziom wierzchołka}
Zadania typu klasyfikacja wierzchołka, regresja wierzchołka
\subsection{Poziom krawędzi}
Zadania typu predykcja połączeń, klasyfikacja połączeń
\subsection{Poziom grafu}
Zadania typu klasyfikacja grafów

\section{Rodzaje warunków treningowych i zastosowania}
\subsection{Uczenie nadzorowane i półnadzorowane}
\subsection{Uczenie nienadzorowane}
\subsection{}

\begin{itemize}

  \item Grafy skierowane vs nieskierowane
  \item Grafy homogeniczne vs heterogeniczne
  \item Grafy dynamiczne vs statyczne
  \item Zastosowania grafów (node classification, edge classification, node
        prediction, edge prediction, node embedding etc.)
  \item Rodzaje uczenia (supervised, semi-supervised, unsupervised)

\end{itemize}

\section{Sztuczne sieci neuronowe}
\section{Grafowe sieci neuronowe (GNN)}
\section{GraphSAGE}


\section{Mini-batching danych treningowych z grafu}



\chapter{Przegląd literatury}

\chapter{Konkrety}
\section{Dane}
\subsection{Źródło danych 1 - Reddit dataset}
\subsection{Źródło danych 2 - Czech bank data}

\section{Metody i sposób badania}
\subsection{Metryki użyte do porównania skuteczności}
\section{Wyniki}
\section{Możliwe kierunki rozwoju badania}



\chapter{Wnioski}



\clearpage
\addcontentsline{toc}{chapter}{Bibliografia}

\bibliographystyle{plain}
\bibliography{refs}
\clearpage
\addcontentsline{toc}{chapter}{Spis rysunków}
\listoffigures

\clearpage
\listoftables
\addcontentsline{toc}{chapter}{Spis tabel}

\appendix
\chapter*{Kody źródłowe}
\addcontentsline{toc}{chapter}{Kody źródłowe}

\chapter*{Streszczenie}
\addcontentsline{toc}{chapter}{Streszczenie}
Streszczenie między 250-300 słów/ ok 900 znaków

\clearpage



\end{document}
